\documentclass{article}

\usepackage{fancyhdr}
\usepackage{extramarks}
\usepackage{amsmath}
\usepackage{amsthm}
\usepackage{amsfonts}
\usepackage{tikz}
\usepackage[plain]{algorithm}
\usepackage{algpseudocode}
\usepackage{listings}
\usepackage{enumerate}
\lstset{breaklines=true}

\usetikzlibrary{automata,positioning}

%
% Basic Document Settings
%

\topmargin=-0.45in
\evensidemargin=0in
\oddsidemargin=0in
\textwidth=6.5in
\textheight=9.0in
\headsep=0.25in

\linespread{1.1}

\pagestyle{fancy}
\lhead{\hmwkAuthorName}
\chead{\hmwkClass\ (\hmwkClassInstructor): \hmwkTitle}
\rhead{\firstxmark}
\lfoot{\lastxmark}
\cfoot{\thepage}

\renewcommand\headrulewidth{0.4pt}
\renewcommand\footrulewidth{0.4pt}

\setlength\parindent{0pt}

%
% Create Problem Sections
%

\newcommand{\enterProblemHeader}[1]{
    \nobreak\extramarks{}{Problem \arabic{#1} continued on next page\ldots}\nobreak{}
    \nobreak\extramarks{Problem \arabic{#1} (continued)}{Problem \arabic{#1} continued on next page\ldots}\nobreak{}
}

\newcommand{\exitProblemHeader}[1]{
    \nobreak\extramarks{Problem \arabic{#1} (continued)}{Problem \arabic{#1} continued on next page\ldots}\nobreak{}
    \stepcounter{#1}
    \nobreak\extramarks{Problem \arabic{#1}}{}\nobreak{}
}

\setcounter{secnumdepth}{0}
\newcounter{partCounter}
\newcounter{homeworkProblemCounter}
\setcounter{homeworkProblemCounter}{1}
\nobreak\extramarks{Problem \arabic{homeworkProblemCounter}}{}\nobreak{}

%
% Homework Problem Environment
%
% This environment takes an optional argument. When given, it will adjust the
% problem counter. This is useful for when the problems given for your
% assignment aren't sequential. See the last 3 problems of this template for an
% example.
%
\newenvironment{homeworkProblem}[1][-1]{
    \ifnum#1>0
        \setcounter{homeworkProblemCounter}{#1}
    \fi
    \section{Problem \arabic{homeworkProblemCounter}}
    \setcounter{partCounter}{1}
    \enterProblemHeader{homeworkProblemCounter}
}{
    \exitProblemHeader{homeworkProblemCounter}
}

%
% Homework Details
%   - Title
%   - Due date
%   - Class
%   - Section/Time
%   - Instructor
%   - Author
%

\newcommand{\hmwkTitle}{Homework\ \#1}
\newcommand{\hmwkDueDate}{April 14, 2015}
\newcommand{\hmwkClass}{Web Information Management}
\newcommand{\hmwkClassTime}{TTh 12:10}
\newcommand{\hmwkClassInstructor}{Professor Fang}
\newcommand{\hmwkAuthorName}{Rick Sullivan}

%
% Title Page
%

\title{
    \vspace{2in}
    \textmd{\textbf{\hmwkClass:\ \hmwkTitle}}\\
    \normalsize\vspace{0.1in}\small{Due\ on\ \hmwkDueDate}\\
    \vspace{0.1in}\large{\textit{\hmwkClassInstructor\ \hmwkClassTime}}
    \vspace{3in}
}

\author{\textbf{\hmwkAuthorName}}
\date{}

\renewcommand{\part}[1]{\textbf{\large Part \Alph{partCounter}}\stepcounter{partCounter}\\}

%
% Various Helper Commands
%

% Useful for algorithms
\newcommand{\alg}[1]{\textsc{\bfseries \footnotesize #1}}

% For derivatives
\newcommand{\deriv}[1]{\frac{\mathrm{d}}{\mathrm{d}x} (#1)}

% For partial derivatives
\newcommand{\pderiv}[2]{\frac{\partial}{\partial #1} (#2)}

% Integral dx
\newcommand{\dx}{\mathrm{d}x}

% Alias for the Solution section header
\newcommand{\solution}{\textbf{\large Solution}}

% Probability commands: Expectation, Variance, Covariance, Bias
\newcommand{\E}{\mathrm{E}}
\newcommand{\Var}{\mathrm{Var}}
\newcommand{\Cov}{\mathrm{Cov}}
\newcommand{\Bias}{\mathrm{Bias}}

\begin{document}

\maketitle

\pagebreak

\begin{homeworkProblem}
    Assume that a search engine returns a ranked list of 10 total documents for a given query.
    According to the ground truth labeling, there are 7 relevant documents for this query, and that the
    relevant documents in the ranked list are in the 1st, 3rd, 5th, 8th, and 10th positions in the ranked
    results.
    \begin{enumerate}
        \item Calculate Precision, Recall, F-measure, nDCG, at the 10 retrived documents.
        \item Calculate the interpolated precision value for each of the following standard recall levels:
        \[\{0.0, 0.1, 0.2, 0.3, 0.4, 0.5, 0.6, 0.7, 0.8, 0.9, 1.0\}\] for this individual query.
        \item Calculate the average precision.
    \end{enumerate}

    \textbf{Part 1}
    \\
    Precision is determined by \[\frac{\#\ of\ returned\ relevant\ documents}{total\ \#\ of\ returned\ documents}.\]
    We have 5 relevant documents returned, and 10 total returned documents, so \(P=0.5=50\%\).
    \\

    Recall is determined by \[\frac{\#\ of\ returned\ relevant\ documents}{total\ \#\ of\ relevant\ documents}.\]
    We have 5 relevant documents returned, and 7 total relevant documents, so \(R=5/7\approx0.714=71.4\%\).
    \\

    F-measure is the harmonic mean of recall and precision, 
    \[
    \begin{split}
        F &= \frac{2PR}{P+R}\\
        &=  \frac{2(0.5)(0.714)}{0.5+0.714}\\
        &\approx 0.5881
    \end{split}
    \]
    \\

    nDCG
    \\

    The discounted gain list
    \[1, 0, 1/\log_2(3), 0, 1/\log_2(5), 0, 0, 1/\log_2(8), 0, 1/\log_2(10)\]

    gives DCG
    \[1, 1, 1.63, 1.63, 2.06, 2.06, 2.06, 2.39, 2.39, 2.69\]

    The ideal DCG is 
    \[1, 2, 2.63, 3.13, 3.56, 3.95, 4.30, 4.30, 4.30, 4.30\]

    So the nDCG is 
    \[1, 0.5, 0.62, 0.52, 0.58, 0.52, 0.48, 0.56, 0.56, 0.56\]

    \textbf{Part 2}
    \\

    Precision after \(n\) documents:
    \[1/1, 1/2, 2/3, 2/4, 3/5, 3/6, 3/7, 4/8, 4/9, 5/10\]
    
    Recall after \(n\) documents:
    \[1/7, 1/7, 2/7, 2/7, 3/7, 3/7, 3/7, 4/7, 4/7, 5/7\]

    So the interpolated precisions are
    \\

    \begin{tabular}{c c}
        Recall level & interpolated precision\\\hline
        0.0 & 1\\
        0.1 & 1\\
        0.2 & 0.67\\
        0.3 & 0.6\\
        0.4 & 0.6\\
        0.5 & 0.5\\
        0.6 & 0.5\\
        0.7 & 0.5\\
        0.8 & 0\\
        0.9 & 0\\
        1.0 & 0\\
    \end{tabular}
    \\

    \textbf{Part 3}
    \\

    The average precision is the average of the list in part 2, which gives
    \[ P_{avg} = 56.4\%\]
\end{homeworkProblem}

\begin{homeworkProblem}
    The \textit{San Jose Mercury News} repository from 2000 to 2005 (i.e., 5 years) contains about 400
    million word tokens, with the vocabulary size about 1 million. What would be a good estimation
    of the vocabulary size one would get in indexing the \textit{San Jose Mercury News} repository from
    2000 to 2010 (i.e., 10 years)?
    \\

    \textbf{Solution}
    \\

    \[
        \begin{split}
        V_R(n) &= Kn^\beta\\
        V_R(400,000,000) &= K(400,000,000)^\beta = 1,000,000
        \end{split}
    \]
    \\

    \(\beta\) is usually between 0.4 and 0.6. Let's choose \(\beta=0.5\).
    This means 

    \[
        \begin{split}
        K(400,000,000)^{0.5} &= 1,000,000\\
        K &= 50
        \end{split}
    \]

    Therefore, for double the amount of word tokens, we have

    \[
        \begin{split}
            V_R(800,000,000) &= 50(800,000,000)^0.5 \approx 1,414,213
        \end{split}
    \]
\end{homeworkProblem}

\begin{homeworkProblem}
    Assuming Zipf’s law with a corpus independent constant \(A = 0.1\), what is the fewest number of most frequent words that together account for more than 22\% of word occurrences (i.e. the minimum value of m such that at least 22\% of word occurrences are one of the m most frequent words).
    \\

    \textbf{Solution}
    \\

    Given for Zipf's law
    \[
        f * r = 0.1
    \]

    we want to find the smallest \(n\) such that 
    \[
        \sum_{i=0}^{n} 0.1/r_i \geq 0.22
    \]

    Which is the series \([0.1, 0.15, 0.1833, 0.2083, 0.2283]\).
    The fewest number of most frequest words that account for 22\% of all word occurrences is 5 words.
\end{homeworkProblem}

\begin{homeworkProblem}
    Assume the following documents comprise your corpus:
    \\

    Doc 1: banking on banks to raise the interest rate over the previous interest
    \\
    Doc 2: jogging along the river bank to look at the sailboats
    \\
    Doc 3: jogging to the bank to look at the interest rate
    \\
    Doc 4: buzzer-beating shot banked in!
    \\
    Doc 5: interest of the scenic outlooks on the banks of the Potomac River
    \\

    Assume that you remove stopwords, lower cases, and do stemming.
    \\

    \begin{enumerate}[1.]
        \item Unranked Boolean retrieval with the following query:
            \\
            (bank OR rate) AND interest

        \item Ranked Boolean retrival with the following query:
            \\
            (bank OR rate) AND interest
    \end{enumerate}

    \textbf{Solution}
    \\

    Stemmed word occurrences
    \\

    \begin{tabular}{c c c c c c }
        & \multicolumn{5}{c}{Documents}\\
        & 1 & 2 & 3 & 4 & 5\\\hline
        bank & 2 & 1 & 1 & 1 & 1\\
        rate & 1 & 0 & 1 & 0 & 0\\
        interest & 2 & 0 & 1 & 0 & 1
    \end{tabular}
    \\

    \textbf{Part 1}
    \\
Documents 1, 3, and 5 would be returned. All documents contain bank and/or rate, so the documents that also have interest match the query.
    \\

    \textbf{Part 2}
    \\
    \begin{tabular}{c c c c c c }
        & \multicolumn{5}{c}{Documents}\\
        & 1 & 2 & 3 & 4 & 5\\\hline
        bank OR rate & 3 & 1 & 2 & 1 & 1\\
        (bank OR rate) AND interest & 2 & 0 & 1 & 0 & 1\\
    \end{tabular}
    \\

    Documents would be returned in the following order: 1, 3, 5 (3 and 5 could be in either order). The OR operation adds occurrence counts, while the AND operation takes the minimum value of either option.
\end{homeworkProblem}
\begin{homeworkProblem}
    As we discussed in class, stemming can generally help improve retrieval performance in English
    text, but it may not always be the case. Choose one of the three choices: ``stemming'' or ``no
    stemming'' or ``maybe stemming'', for each of the following 4 cases. You need to provide the
    reasons/explanations for each of your choices.
    \\

    \textbf{Solution}
    \\
    \begin{enumerate}[1.]
        \item When the corpus is small and users care more about precision
            \\

            No stemming. Stemming reduces resources required to maintain and return relevant information (by reducing vocabulary size), which is not a problem with a small corpus. Also, stemming attempts to simplify the user's query, which can reduce the precision of queries, as multiple words may stem to the same base.
        \item When the corpus is small and users care more about recall
            \\

            Stemming (or at least probably stemming). Stemming words will make sure that almost \textit{any} relevant document will be returned, even if precision is reduced. Small corpuses (corpi?) will not benefit from stemming in terms of resources, but increased recall may be beneficial.
        \item When the corpus is large and users care more about precision
            \\

            Maybe stemming. Stemming may give some resource benefits by reducing vocabulary size, but it comes at the cost of precision. Having stemming will increase precision.

        \item When the corpus is large and users care more about recall
            \\

            Stemming. As mentioned earlier, stemming can both increase recall and improve performance with a large corpus.
    \end{enumerate}
\end{homeworkProblem}

%\pagebreak
%\section{Appendix}

\end{document}
