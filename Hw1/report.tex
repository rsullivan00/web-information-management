\documentclass{article}

\usepackage{fancyhdr}
\usepackage{extramarks}
\usepackage{amsmath}
\usepackage{amsthm}
\usepackage{amsfonts}
\usepackage{tikz}
\usepackage[plain]{algorithm}
\usepackage{algpseudocode}
\usepackage{listings}
\usepackage{enumerate}
\lstset{breaklines=true}

\usetikzlibrary{automata,positioning}

%
% Basic Document Settings
%

\topmargin=-0.45in
\evensidemargin=0in
\oddsidemargin=0in
\textwidth=6.5in
\textheight=9.0in
\headsep=0.25in

\linespread{1.1}

\pagestyle{fancy}
\lhead{\hmwkAuthorName}
\chead{\hmwkClass\ (\hmwkClassInstructor): \hmwkTitle}
\rhead{\firstxmark}
\lfoot{\lastxmark}
\cfoot{\thepage}

\renewcommand\headrulewidth{0.4pt}
\renewcommand\footrulewidth{0.4pt}

\setlength\parindent{0pt}

%
% Create Problem Sections
%

\newcommand{\enterProblemHeader}[1]{
    \nobreak\extramarks{}{Problem \arabic{#1} continued on next page\ldots}\nobreak{}
    \nobreak\extramarks{Problem \arabic{#1} (continued)}{Problem \arabic{#1} continued on next page\ldots}\nobreak{}
}

\newcommand{\exitProblemHeader}[1]{
    \nobreak\extramarks{Problem \arabic{#1} (continued)}{Problem \arabic{#1} continued on next page\ldots}\nobreak{}
    \stepcounter{#1}
    \nobreak\extramarks{Problem \arabic{#1}}{}\nobreak{}
}

\setcounter{secnumdepth}{0}
\newcounter{partCounter}
\newcounter{homeworkProblemCounter}
\setcounter{homeworkProblemCounter}{1}
\nobreak\extramarks{Problem \arabic{homeworkProblemCounter}}{}\nobreak{}

%
% Homework Problem Environment
%
% This environment takes an optional argument. When given, it will adjust the
% problem counter. This is useful for when the problems given for your
% assignment aren't sequential. See the last 3 problems of this template for an
% example.
%
\newenvironment{homeworkProblem}[1][-1]{
    \ifnum#1>0
        \setcounter{homeworkProblemCounter}{#1}
    \fi
    \section{Problem \arabic{homeworkProblemCounter}}
    \setcounter{partCounter}{1}
    \enterProblemHeader{homeworkProblemCounter}
}{
    \exitProblemHeader{homeworkProblemCounter}
}

%
% Homework Details
%   - Title
%   - Due date
%   - Class
%   - Section/Time
%   - Instructor
%   - Author
%

\newcommand{\hmwkTitle}{Homework\ \#1}
\newcommand{\hmwkDueDate}{April 14, 2015}
\newcommand{\hmwkClass}{Web Information Management}
\newcommand{\hmwkClassTime}{TTh 12:10}
\newcommand{\hmwkClassInstructor}{Professor Fang}
\newcommand{\hmwkAuthorName}{Rick Sullivan}

%
% Title Page
%

\title{
    \vspace{2in}
    \textmd{\textbf{\hmwkClass:\ \hmwkTitle}}\\
    \normalsize\vspace{0.1in}\small{Due\ on\ \hmwkDueDate}\\
    \vspace{0.1in}\large{\textit{\hmwkClassInstructor\ \hmwkClassTime}}
    \vspace{3in}
}

\author{\textbf{\hmwkAuthorName}}
\date{}

\renewcommand{\part}[1]{\textbf{\large Part \Alph{partCounter}}\stepcounter{partCounter}\\}

%
% Various Helper Commands
%

% Useful for algorithms
\newcommand{\alg}[1]{\textsc{\bfseries \footnotesize #1}}

% For derivatives
\newcommand{\deriv}[1]{\frac{\mathrm{d}}{\mathrm{d}x} (#1)}

% For partial derivatives
\newcommand{\pderiv}[2]{\frac{\partial}{\partial #1} (#2)}

% Integral dx
\newcommand{\dx}{\mathrm{d}x}

% Alias for the Solution section header
\newcommand{\solution}{\textbf{\large Solution}}

% Probability commands: Expectation, Variance, Covariance, Bias
\newcommand{\E}{\mathrm{E}}
\newcommand{\Var}{\mathrm{Var}}
\newcommand{\Cov}{\mathrm{Cov}}
\newcommand{\Bias}{\mathrm{Bias}}

\begin{document}

\maketitle

\pagebreak

\begin{homeworkProblem}
    Assume that a search engine returns a ranked list of 10 total documents for a given query.
    According to the ground truth labeling, there are 7 relevant documents for this query, and that the
    relevant documents in the ranked list are in the 1st, 3rd, 5th, 8th, and 10th positions in the ranked
    results.
    \begin{enumerate}
        \item Calculate Precision, Recall, F-measure, nDCG, at the 10 retrived documents.
        \item Calculate the interpolated precision value for each of the following standard recall levels:
        \[\{0.0, 0.1, 0.2, 0.3, 0.4, 0.5, 0.6, 0.7, 0.8, 0.9, 1.0\}\] for this individual query.
        \item Calculate the average precision.
    \end{enumerate}

    \textbf{Part 1}
    \\
    Precision is determined by \[\frac{\#\ of\ returned\ relevant\ documents}{total\ \#\ of\ returned\ documents}.\]
    We have 5 relevant documents returned, and 10 total returned documents, so \(P=0.5=50\%\).
    \\

    Recall is determined by \[\frac{\#\ of\ returned\ relevant\ documents}{total\ \#\ of\ relevant\ documents}.\]
    We have 5 relevant documents returned, and 7 total relevant documents, so \(R=5/7\approx0.714=71.4\%\).
    \\

    F-measure is the harmonic mean of recall and precision, 
    \[
    \begin{split}
        F &= \frac{2PR}{P+R}\\
        &=  \frac{2(0.5)(0.714)}{0.5+0.714}\\
        &\approx 0.5881
    \end{split}
    \]
    \\

    nDCG
    \\
\end{homeworkProblem}

\begin{homeworkProblem}
    The \textit{San Jose Mercury News} repository from 2000 to 2005 (i.e., 5 years) contains about 400
    million word tokens, with the vocabulary size about 1 million. What would be a good estimation
    of the vocabulary size one would get in indexing the \textit{San Jose Mercury News} repository from
    2000 to 2010 (i.e., 10 years)?
    \\

    \textbf{Solution}
    \\

\end{homeworkProblem}

\begin{homeworkProblem}
    Assume the following documents comprise your corpus:
    \\
    Doc 1: banking on banks to raise the interest rate over the previous interest
    \\
    Doc 2: jogging along the river bank to look at the sailboats
    \\
    Doc 3: jogging to the bank to look at the interest rate
    \\
    Doc 4: buzzer-beating shot banked in!
    \\
    Doc 5: interest of the scenic outlooks on the banks of the Potomac River
    \\
    Assume that you remove stopwords, lower cases, and do stemming.
    \\
\end{homeworkProblem}

%\pagebreak
%\section{Appendix}

\end{document}
